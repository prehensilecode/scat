\documentclass[10pt,titlepage,times,letterpaper]{article}
\usepackage{enumerate}
\usepackage{epsf}                
\usepackage{natbib}
\usepackage{graphicx}
\newtheorem{algorithm}{Algorithm}[section]
\usepackage{amsmath}
\usepackage{color}
%define colors
\definecolor{mediumblue}{rgb}{0.0509,0.35,0.568}
\definecolor{blue}{rgb}{0.0109,0.15,0.468}
\definecolor{lightblue}{rgb}{0.81176,0.92549,0.9333}
\definecolor{yellow}{rgb}{0.961,0.972,0.047}
\definecolor{red}{rgb}{0.9,0.1,0.1}
\definecolor{orange}{rgb}{1.0,0.4,0.0}
\definecolor{darkblue}{rgb}{0.109,0.35,0.538}


\def\SCAT{{\tt SCAT2} }
\def\Di{{\cal D}}

\def \Pr{{\rm Pr}}
\def \E{{\rm E}}
\def \Var{{\rm Var}}
\def \rhobar{{\bar{\rho}}}

\pagenumbering{arabic}
\begin{document}

\title{Documentation for {\tt SCAT}, version 2.3}

\author{
Software code and documentation by Matthew Stephens\footnote{email: {\tt mstephens@uchicago.edu}} \\
WWW: {\tt http://github.com/stephens999/scat}\\
\\
}

\date{September 28 2004; updated September 15 2015; revised by Mary Kuhner\footnote{email:  {\tt mkkuhner@uw.edu}} July 5 2021}


\maketitle

\tableofcontents
\vfil\eject
\section{Introduction}

The program \SCAT (Smoothed and Continuous AssignmenTs)
implements a Bayesian statistical method for estimating allele
frequencies and assigning samples of unknown (or known) origin across
a continuous range of locations, based on genotypes collected at
distinct sampling locations.  In brief, the idea is to assume that allele
frequencies vary smoothly in the study region, so allele frequencies
are estimated at any given location using observed genotypes at
nearby sampling locations, with data at the nearest sampling
locations being given greatest weight.  The software can deal with SNP
and microsatellite and other multi-allelic loci, in any combination,
and missing data are allowed. Details of the method are given in
Wasser et al. (2004).

The program is fairly slow:  data sets attempting to locate a thousand
samples can take 3-4 days to run on a moderately fast machine.  We
recommend trial runs with fewer samples to get an idea of the time and
space requirements.

This document describes the use of this software, which comes free of
charge, and with no warranty whatsoever.  Updates for the software
will be made available via
$$\text{{\tt http://github.com/stephens999/scat}}$$
Please send bug reports and requests for new features by opening an issue on that page.

In publications including results from the use of this program, please
{\it specify the version of the software you used}, and cite
Wasser et al. (2004).



%%%%%%%%%%%%%%%%%%%%%%%%%%%%%%%%%%%%%%%%%%%%%%%%

\section{Getting started: installing and running the software} \label{started}

\SCAT was previously distributed as executables for MAC OSX and
Linux, but currently only for Linux.  Users of other operating systems
will need to compile from source.

The Linux executable comes in a gzipped tar file.
Extract the files from the {\tt .tar.gz} file by typing
$$\text{{\tt gunzip scat.linux.xx.tar.gz}}$$
$$\text{{\tt tar -xvf scat.linux.xx.tar}}$$
at the command line, where {\tt xx} is the version number.
This will create a new directory called something like
{\tt scat.linux.xx}. Change to this directory before
running the program.

Source code is available on the github page. 
To build the program from source, unpack the source code into a directory and type:

{\tt make}

The Makefile as distributed makes an optimized version of the program for speed.
If you need to run the program in a debugger such as {\tt gdb}, turn off optimization 
and turn on debugging support by locating the following line in the Makefile:

\begin{verbatim}
CFLAGS = -O3 -std= -DNDEBUG $(INCLUDE)
\end{verbatim}

and inserting a hash (\#) at the beginning, and locating the following line:

\begin{verbatim}
#CFLAGS = -g $(INCLUDE)
\end{verbatim}

and removing its hash.  Then type {\tt make clean} and {\tt make}.   
The resulting code will be much slower and we recommend this only if you 
are using a debugger.  It will also do extensive checking for internal
consistency; if the program runs in optimized mode but aborts in 
debug mode, we would very much appreciate a bug report.

The program can be used in two different ways:
\begin{itemize}
\item To estimate allele frequencies at specified
locations, based on genotypes observed at a number of sampling
locations.
\item To ``assign'' (estimate the location of origin of)
individuals based on allele
frequencies estimated using reference samples from a number of
sampling locations.
\end{itemize}

In this section we will illustrate allele frequency estimation.
See section \ref{assign} for
instructions on using the program for assignment.

To run the program, you must supply two input files: a genotype file,
containing the genotype information, and a location file, containing
information on the latitude and longitude of sampling locations. To
perform continuous assignment tests you must also supply a file giving the
boundaries of the habitat of the individuals being assigned - see
section \ref{boundary}.  Instructions for how to prepare these files
are given below. Finally, you must create a directory or folder to hold your results.

Once these files are created you can run the program
using
$$\text{{\tt ./SCAT2 <genotype file> <location file> <outputdir> L}}$$ 
where {\tt L} is the number of loci in the genotype file and <outputdir>
is the path to your output directory.

An example genotype file {\tt test.genotype.txt}
and an example location file {\tt test.location.txt} are supplied with
the software.  These have $L=2$.  You can run the program on the test data supplied by placing
the \SCAT executable and the two input files in the current directory and typing:

{\flushleft
\tt{mkdir results}  \\
{\tt ./SCAT2\_1 test.genotype.txt test.location.txt results 2} \\
}

\medskip

The program will input the data, perform a number of iterations, and
output results in files in the results directory you created. 
Section \ref{interpret} describes the output files in more detail.

\section{Analysing your own data}

To analyse your own data, you must prepare a genotype file and
location file in the appropriate format, as described below, and then
run the program as above (replacing 2 with the 
the number of loci in your genotype file). A number of additional options, which
can be used to control certain aspects of how the program runs, are
described in subsequent sections and the appendix.

\subsection{Genotype file format} \label{inputfile}

The genotype file is supplied by the user to specify genotypes of the
individuals to be analysed. The file should be a plain text (.txt) file.
Its format is similar to that
used by the program {\tt STRUCTURE} (see section \ref{software})
and can be represented as follows:

\begin{verbatim}
ID(1) Location(1) Allele(111) Allele(121) Allele(131) ... Allele(1L1)
ID(1) Location(1) Allele(112) Allele(122) Allele(132) ... Allele(1L2)
ID(2) Location(2) Allele(211) Allele(221) Allele(231) ... Allele(2L1)
ID(2) Location(2) Allele(212) Allele(222) Allele(232) ... Allele(2L2)
...
...
ID(N) Location(N) Allele(N11) Allele(N21) Allele(N31) ... Allele(NL1)
ID(N) Location(N) Allele(N12) Allele(N22) Allele(N32) ... Allele(NL2)
\end{verbatim}
where
\begin{enumerate}
\item{\tt ID(i)} is a string, giving a label for individual $i$.
\item{\tt Location(i)} is an integer specifying the ``Location number'' of
the sampling location
of individual $i$ (the location number for each location 
is specified in the location file; see below). Individuals of unknown
origin should be given a location number of -1.
\item{{\tt Allele(il1),Allele(il2)}} are the two alleles in
individual $i$ at locus $l$. {\bf Missing alleles should
be represented by {\tt -999}.}
\end{enumerate}
See the example genotype file, {\tt test.genotype.txt}, for an illustration.

The {\tt -C} option can be used to skip extraneous columns in this
file.  It takes an argument of the number of columns to skip;
columns will be skipped starting with the first column after the location
number.

Some things to note:
\begin{enumerate}
\item There should be no commas separating columns, just white space.
\item Individual IDs must not contain any spaces.
\item Any non-negative integer can be used to denote the alleles at a locus.
For SNP alleles the obvious choice is 0 and 1, but any two numbers can be used,
and it need not be the same two numbers in each column (so recoding SNP data 
using A=1, C=2, G=3 and T=4 is fine).  For microsatellites you can use 
allele length, or number of repeats, or indeed any set of integers.  No use 
is made of the relative values of these numbers; they are arbitrary labels.
However, negative numbers other than {\tt -999} (which will be interpreted as
missing data) should not be used.
\item The software assumes independence of loci (no linkage disequilibrium),
so it is advisable to avoid markers that are very closely linked. 
Use of closely linked markers will overstate the amount of independent 
information available and lead to too-tight estimates of confidence.
\end{enumerate}

\subsection{Location file format}

The location file is supplied by the user to specify the latitude and
longitude of sampling locations, and also of any other locations
for which the user wishes to estimate allele frequencies.
The file has one row for each location, and its format 
can be represented as follows:
\begin{verbatim}
LocId(1)  LocNo(1)  Latitude(1)  Longitude(1)
LocId(2)  LocNo(2)  Latitude(2)  Longitude(2)
...
LocId(S)  LocNo(S)  Latitude(S)  Longitude(S)
\end{verbatim}
where
\begin{enumerate}
\item{\tt LocId(s)} is a string, with no internal spaces or tabs, giving a label for location $s$
($s = 1,\dots,S$).
\item{\tt LocNo(s)} is an integer specifying the ``Location
number'' of sampling location $s$. These location numbers can be
any $S$ distinct integers. Typically they will simply increase from $1$ to $S$
(so the location number of sampling location $s$ will be $s$). 
\item{{\tt Latitude(s) Longitude(s)}} are the {\it decimal} latitude
and longitude of location $s$. Note that N and E are entered as
positive numbers, and S and W are entered as negative. So, for
example, $47^\circ 39' N$ and $122^\circ 18' W$ become 47.65 and -122.30.
\end{enumerate}
See the example location file {\tt test.location.txt} for an illustration.

If the -Z (Sectors) option is in effect, an additional column must be added to
the location file:

\begin{small}
\begin{verbatim}
LocId(1)  LocNo(1)  SectorNo(1)  Latitude(1)  Longitude(1)
\end{verbatim}
\end{small}


\subsection{Estimating allele frequencies}

To estimate allele frequencies at all locations in the location
file, simply run the program as in section \ref{started} above,
using
$$\text{{\tt ./SCAT2 <genotype file> <location file> <outputdir> L}}$$

\subsection{Assigning individuals}  \label{assign}

To estimate the location of individuals (``assignment") you will need
to set either {\tt -A} or {\tt -M} (the ``assignment option").  The
only difference between these options is that {\tt -A} takes the
individuals to be assigned from the main genotype file, and {\tt -M}
takes them from a separate file.  One or the other may be more
convenient for your pipeline.  

Both assignment options perform two types of assignments: {\it smoothing
 assignment,} which assigns individuals to existing locations, and 
{\it continuous assignment,} which
assigns them to geographical coordinates that may not coincide with any
existing location. 

WARNING:  If you set neither assignment option, smoothing assignments
will still be produced; but no cross-validation will be done, and the certainty of
the assignments will be inflated as a result.

Smoothing assignment uses allele frequencies estimated at each
location in the location file to compute the probability of each
individual's genotype in each location, and outputs the results in the
{\tt \_probs} file (see section \ref{out}). Note that these
probabilities are also computed even if no assignment option is
used; the difference is that when an assignment option is invoked these
probabilities are computed using leave-one-out cross-validation (ie
the genotypes of individual $i$ are ignored when estimating allele
frequencies for assigning individual $i$).

Continuous assignment uses allele frequency estimates to assign
individuals, allowing an individual's location of origin to be
anywhere in the organism's range (in particular, the location may
be somewhere other than any of the locations in the location
file). The organism's range should be specified using a
boundary file, as described below. The software outputs a number of
estimates for the possible location of each individual, one for each
iteration -- both burn-in and main iterations -- of the MCMC
scheme. The results for the burn-in iterations (the first 100 values
under the default parameter, since {\tt Nburn=100} by default - see section
\ref{HowLong}) should generally be discarded. The median latitude and
longitude of the remaining samples can be used as a point estimate for
the individual's location, and the spread of the points gives an
indication of the uncertainty. The results for each individual are saved 
in a file whose name is the same as the individual ID. (The third column in
this file gives a measure of the log likelihood, which can be
used to investigate mixing.)

\subsubsection{-A option}

The {\tt -A} option is used when
individuals to be assigned are included in the genotype file, on
consecutive lines. Individuals of unknown origin should be given a
location number of -1 in the genotype file.  If an individual is marked for
assignment but given a location other than -1, it will participate in
allele frequency estimation for that location except for the purposes
of its own assignment, where its frequency contribution will be dropped.

To perform assignments for individuals $i$ to $j$ in the input file,
simply add {\tt -A i j} after the \SCAT command. For example, to
perform assignments on the first, second and third individuals in the
genotye file use

{\tt ./SCAT2\_1 -A 1 3 <genotype file> <location file> <outputdir> L}

\subsubsection{{\tt -M} option} \label{assign2}

This option behaves exactly like {\tt -A}, except that rather than being
included in the input file, the individuals to be assigned are in a
separate file, whose name you specify straight after the {\tt M}, with
no space between {\tt M} and the filename:
 {\tt -Massignmentfile.txt} 

The {\tt -M} option does not take arguments; it will always assign all
individuals in the assignment file.
 
\subsection{Specifying the organism's range} \label{boundary}

When performing continuous assignments it is {\it highly}
recommended that you specify the range within which individuals might
be found. Without this individuals may be assigned to unrealistic
locations (eg land-dwelling organisms assigned to ocean).  If only
smoothing assignment is needed, this is not necessary as individuals will
only be assigned to sampling locations.

Two methods of specifying the organism's range are provided:  as
a grid indicating all 1 degree squares within the area, or as a polygon
bounding the area.  The former is recommended, and {\it must} be
used if the \SCAT data are intended as input to the VORONOI program.  (The
organism's range must be the same in \SCAT and VORONOI, and VORONOI cannot
read the polygon boundary format.) 

Not all organism ranges can be gracefully indicated by the current code:
see section \ref{cautions}.  In particular, the range cannot include either pole
or cross the line opposite the Prime Meridian.

\subsubsection{Grid file:  {\tt -g} option}

The organism's range can be specified by dividing the area of interest into grid
squares of 1 degree latitude and longitude, and listing all grid squares which could
contain the organism.  An advantage of this method is that it allows a discontinuous
habitat, for example an island organism found on several islands but not in the ocean.
If the organism has a very large 
north-south range the distortion of the 1 degree grid squares may be problematic.

To specify the habitat in this way, prepare a file which contains one line per
square included in the habitat, giving decimal latitude and decimal longitude with
a space between them.  The latitude and longitude describe the lower left corner of
the grid square, so that an entry:

{\tt -12 17}

indicates the grid square whose lower left corner is 12 S and 17 E, and whose upper right 
corner is 11 S and 18 E.

The grid file is then specified to the program using the {\tt -g} option followed
by the filename of the grid file:

{\tt -g ../savannah\_grid.txt}

Note that there is a space between the option and the filename.

\subsubsection{Boundary file:  {\tt -B} option}
The region can be specified as a polygon by entering the latitude and longitude of
each consecutive vertex into a ``boundary'' file (see below), and using
the {\tt -B} option to indicate the name of this file.
For example, if the boundary file is called {\tt eg.boundary.txt} you
should run \SCAT using
$$\text{\tt ./SCAT2 -Beg.boundary.txt ...}$$ 

Note that there is no space between the B and the file name.
The boundary file should contain one row for each vertex in the
polygon, with two numbers on each row (decimal latitude and longitude,
with N and E being positive, S and W being negative) separated by a
space. The vertices should be entered in order (in either direction
around the polygon), with coordinates of the last vertex being an {\it
exact} repeat of the coordinates of the first vertex.

For example, to specify a square region, from $5^\circ N$ to $3^\circ S$ and
$2^\circ W$ to $1^\circ E$ the file could be
\begin{verbatim}
5 -2
5  1
-3 1
-3 -2
5 -2
\end{verbatim}

The routine that tests whether a point is inside or outside the polygon
is based on the 2-D algorithm helpfully
described by Dan Sunday ({\tt http://www.softsurfer.com}). 
Limitations of the algorithm are described in section \ref{cautions}.

\subsubsection{Pre-loaded Afrian elephant location data}

The program contains pre-loaded boundaries for savannah and forest 
African elephants.  These boundaries are very loose (in particular
the savannah boundary contains a good deal of ocean) and we do not
recommend their use for any purpose other than replicating previous
results.

\begin{itemize}
\item[{\tt -d}] use hard-coded boundaries for savannah African elephants
\item[{\tt -D}] use hard-coded boundaries for forest African elephants
\end{itemize}

%--------------------------------------------------------------------

\section{Cautions for use of the software} \label{cautions}

\SCAT carries out an MCMC search.  If the search is too short or mixes
poorly, the results may be inaccurate and their certainty will be overstated.  The program
{\tt TRACER} (see section \ref{software}) may be useful in diagnosing poor
searches.

The code which sets the boundaries of the organism's range
will not work correctly if the range crosses the line opposite
the Prime Meridian or includes either pole.  Nothing is done by the code to
diagnose such problems.  
The grid boundary approach may also be problematic if the species has a wide
north-south distribution approaching either pole, as the size of the grid 
squares is assumed to be roughly constant and this will not be the case.
If such a species needs to be analyzed, we recommend rescaling the latitudes and
longitudes to make them tractable, and then reversing the rescaling before
interpreting the data.  

If the species occurs as widely separated patches, using a grid file
to specify correct species boundaries may cause the program to become
stuck on the first solution it finds because other solutions are 
``too far away" in the patchy map.  You can diagnose this by making
multiple runs with different random number seeds:  if each run places a
given individual with great certainty, but different runs place the
individual in different patches, this problem is likely occuring.  Combining
estimates across multiple runs will give a better estimate than any
single run.  Otherwise, it may be necessary to make the map more continuous
even if this requires adding areas where the organism is not found, or to
use smoothing rather than continuous assignment.

\SCAT assumes that markers are unlinked.  Tightly linked markers will
cause the accuracy of the results to be overstated.  Sequencing individual
genes and then using all variable sites within them as markers is likely
to cause this problem and cannot be recommended as a way to obtain \SCAT
input data.

\SCAT assumes that allele frequencies change smoothly with geographic
distance.  If there are strong discontinuities in gene flow, it will
inappropriately use populations which are nearby but separated by the
discontinuity to inform each others' allele frequencies.  It may be
helpful to separate the discontinuous areas into separate \SCAT runs,
as was done for forest and savannah elephants in Wasser et al. (2004).

There is a hard-coded constant for MAXNALLELES, the maximum number
of alleles per locus, near the top of file scat2.hpp.  It is set to 120
by default, which is sufficient for microsatellite data and overkill for
SNP data.  If you have loci with more alleles than this, edit the
file and then type ``make clean" followed by ``make".  If you are using
SNP data it may be helpful to set this constant to 5 (four bases and unknown
data) in order to save memory.

The practice of summarizing \SCAT results as median latitude and
median longitude can be misleading if the individual's
inferred distribution is patchy or irregular in shape.  Displaying
a heatmap of all \SCAT results for that individual may be preferable,
and/or reporting the bounding region containing a given percentage
of the estimates.

In our hands, \SCAT does a poor job localizing individuals which are
hybrids between distant populations or species.  We recommend the use
of {\tt EBhybrids} (see section \ref{software}) to remove putative hybrids.
Our preliminary studies suggest that hybrids in the reference data do
relatively little harm, but the inferred locations of hybrid individuals are
unreliable. 

Occasionally the heatmap of an individual may resemble a thin rim around the
very edge of the organism's range.  In our experience this may mean that
the individual is a hybrid, or that it comes from a location which has been
incorrectly excluded from the range.  In either case, the program finds that
the individual is a poor fit for all sampling locations, so moves it as far
away from all of them as possible, leading to the thin rim.

\section{Search parameters} \label{HowLong}

\subsection{Run length}
\SCAT employs an iterative scheme to perform inference. 
Parameters that control the number of iterations performed
can be added to the input line, as follows:
$$\text{{\small \tt ./SCAT2 <genotype file> <location file> <output
filename> L Niter Nthin Nburn }}$$ where {\tt Niter} is the number
of sampled iterations of the MCMC scheme to be performed, {\tt Nthin} is the
number of steps taken through the Markov chain between samplings (the ``thinning interval") and {\tt
Nburn} is the number of burn-in iterations.

For example, to set
$$\text{{\tt Niter=100}}, \text{{\tt Nthin=10}}, \text{{\tt Nburn=100}}$$
use
$$\text{{\tt \small./SCAT2 <genotype file> <location file> <output
filename> L 100 10 100 }}$$ 
In fact, the above values are the
default values. Each iteration performs {\tt Nthin}
steps through the Markov chain.
The number of iterations required to obtain accurate answers depends
on the complexity and size of the data set; preliminary runs are useful
to establish a good run length.

\subsection{Setting the seed ({\tt S})} \label{Soption}

The {\tt -S} option can be used to set the seed of the
pseudo-random number generator. The seed should be a
positive integer, and must follow the {\tt -S} after a space, as in:
$$\text{{\tt ./SCAT2 -S 3253 ...}}$$
which will set the seed to be 3253. This option can be
used to deliberately duplicate a previous run, or to make sure 
that you do {\it not}
duplicate a previous run.  If the seed is not set, a 
random number seed based on the computer clock will be used;
this can lead to multiple runs getting the same seed if they are started
at almost exactly the same time, for example by a batch program.

\subsection{Isolation by distance parameters ({\tt -f})}

\SCAT uses a model of isolation by distance for the relationship of
different regional populations' allele frequences:  there is expected to
be more similarity between populations that are closer together.  The
behavior of this isolation-by-distance model is controlled by three
$\alpha$ parameters:  $\alpha_0$ controls the degree to which a regional
population varies from the expectation established by other populations,
$\alpha_1$ controls the scaling for the distances, and $\alpha_2$
controls how quickly correlations between populations drop to 0 with distance.
These parameters are estimated by \SCAT during the run.
It also estimates a parameter $\beta$ which governs the variability
of marker loci.  For a more detailed explanation, see Wasser et al. (2004).

It is helpful to do
several initial runs, with different values of the seed for the
random-number-generator (see {\tt -S} option below), to check how many
iterations are necessary for the values of $\alpha$ and $\beta$ to be
reliably estimated across runs. For computational reasons, these
initial runs are best done without setting an assignment option ({\tt -A}
or {\tt -M}).  If there are
substantial differences between parameter estimates, or in the range
of log-likelihoods achieved, in different runs (output in the {\tt
\_params} file) try increasing the lengths of the runs by increasing
either the number of iterations ({\tt Niter}) or the thinning interval
({\tt Nthin}).  The program {\tt TRACER} (see section \ref{software}) can be
helpful here.  If traces of the parameters appear to move sluggishly,
increase {\tt Nthin}; if there are directional trends visible in the
trace over time, increase either {\tt Niter} or {\tt Nthin}; if the
estimated Effective Sample Size is below 200, increase {\tt Niter}.

The results in Wasser et al. (2004) were obtained by 
first applying the method (without performing assignments)
using a relatively large number of iterations (eg
{Niter=100, Nthin=1000, Nburn=100}), and using the results in the {\tt
\_param} file to get estimates of $\alpha$ and $\beta$.  Next,
five independent sets of assignment runs were done with the parameters $\alpha$
and $\beta$ fixed to their estimated values (see {\tt -f} below), and
the default run-length settings (Niter=100, Nthin=10, Nburn=200). This
might be a reasonable general strategy for performing assignments
(although if the computer time is not prohibitive it might be helpful
to do assignment runs with a larger Nthin, and not fix $\alpha$ and
$\beta$). Note that when fixing $\alpha$ and $\beta$ in this way, it
is recommended that you first find estimates for $\alpha$ and $\beta$
from the {\tt \_params} file (e. g. their mean, median or mode after
discarding burnin), and then choose the row of estimates in the {\tt
\_params} file that is closest (in some sense) to these
estimates. This will help ensure that the {\it combination} of
parameter estimates used is somewhat sensible.

You can fix the values of these
parameters by using {\tt -f}, followed by a space, and then the values
to use for $(\alpha_0,\alpha_1,\alpha_2,\beta)$ (four values, each
separated by spaces).

For example,
$$\text{\tt -f 0.43 5300 0.32 2.3}$$ will fix $\alpha_0 = 0.43, \alpha_1 =
5300, \alpha_2 = 0.32, \beta = 2.3$.

You can also modify the aggressiveness of the search across $\alpha$
values using the following option:

{\tt -a sd} set the proposal standard deviation for alpha updates to
{\tt sd} (default 0.4).  A larger {\tt sd} will correspond to a more
wide-ranging search of alpha values.

\subsection{Setting error parameters ({\tt -e}}

The assignment analysis machinery allows for the probabilities
of an erroneous genotype call at one allele
(error term $\delta$) and of amplification of only one allele (error term
$\gamma$).  The defaults are set to the values used in Wasser et al. (2004)
of $\delta = 0.05$ and $\gamma = 0$.  These values can be changed
using {\tt -e delta gamma}.

\section{Options changing the mathematics or search strategy}

These options change the behavior of the MCMC sampler.  Little information
is available about their usefulness, but they
are documented here for completeness.  In the descriptions below, $X$ refers
to the vector of rescaled allele frequencies described in the
Online Supplement of Wasser et al. 2004.  The \SCAT algorithm normally
proposes new values of $X$ one by one, with a variance of 0.5, but
this can be changed using the options below.

\begin{itemize}
\item[{\tt -a sd}] set the proposal standard deviation for alpha updates to
{\tt sd} (default 0.4).  A larger {\tt sd} will correspond to a more
wide-ranging search of alpha values.

\item[{\tt -h sd}] sets the
proposal standard deviation for $X$ updates to {\tt sd} times
$\sqrt{1/\alpha_0}$ (default = 0.5).  A larger {\tt sd} 
will correspond to a 
more wide-ranging search of $X$ values.  The higher the value, the further
the sampler will attempt to move in one step.  Higher values may help
in preventing the sampler from becoming stuck (diagnosed by persistent
differences between the outcome of different runs on the same data), while
lower values may improve acceptance rate (diagnosed by examining the
Output\_accept file).

\item[{\tt -j}] update $X$ for an allele jointly, rather than one at a time.
No information is currently available on whether this is
helpful.

\item[{\tt -N}] includes a ``nugget effect" in the covariance matrix.  
A nugget effect is variation among samples putatively
taken at the same point, either because of spatial structure too fine
for the sampling scheme to capture, or because of measurement error.
No information is currently available on whether this is helpful, but it
may be worth considering particularly if your sample locations are
rather broadly defined and you think within-location population structure
is likely.

\item[{\tt -r}] use Langevin update for $X$.   This 
update attempts to improve sampling performance by using the estimated 
gradient of the local probability density to choose the direction in
which to move.  The Wikipedia article ``Metropolis-adjusted Langevin 
algorithm" describes this algorithm.  No information is currently
available on whether this is helpful, but it may be worth trying if
mixing is poor.

\item[{\tt -R}] remove
all samples from a region when doing assignment of individuals from
that region.  This is a more aggressive form of cross-validation,
but relies on getting enough allele frequency estimates from nearby
regions to fill in for the disregarded frequencies for this region, and
seems unlikely to work unless the data are quite rich and the number of
regions is fairly large.  It may be particularly appropriate if the
data consist of a large number of regions, each with only a few
individuals, so that region-specific empirical allele frequencies rely
heavily on the few available samples in that region.

\item[{\tt -w}] use smoothing only towards the mean, with no spatial component (ie set
$\alpha_2 = 0$).  This asserts that the species has no spatial 
structure, so that all regions other than the one whose frequencies are
currently being calculated contribute equally to that calculation.  This is
unlikely for real biological populations, though could possibly serve as
a comparison point.

\item[{\tt -X}] ``cheat'' by initialising location close to true location in assignment runs.  
This can be used when assigning locations to 
individuals whose location is already known or suspected.  The search will
converge more rapidly if it starts from a mostly-correct solution.
However, if the search is run for too short a time or mixes poorly, it will
return its starting location with excessive confidence.  We do not recommend
the use of this option to generate publishable results.
It could be used diagnostically to
test whether the sampler is simply stuck at a poor solution, or is actively
moving towards a poor solution.
\end{itemize}


%--------------------------------------------

\section{Interpreting the output} \label{interpret}

\subsection{Verbose output: {\tt -v}}

If this option is set, the program dumps its input data into the
file specified (with a space) after {\tt -v}.   This can be useful
in verifying that the data were read correctly.  
As the program continues to run, it keeps you informed of progress
(burnin iterations, main iterations etc). 

If this option is not set, the program provides little feedback while
running.

\subsection{Output files} \label{out}

The program produces a number of output files in the specified output directory,
whose names are of the form {\tt Output\_xxx}. Their
contents are as follows:

\begin{enumerate}[\_params]
\item[{\tt \_freqs}] Contains estimates (posterior means) of allele
frequencies at each location in the location file, for each locus. It also contains,
for comparison, the empirical allele frequencies at that location.

\item[{\tt \_probs}] Contains estimates of the posterior probability:

$\log(\Pr(\text{genotype
data for individual $i$} | \text{$i$ came from location $j$}))$ 

for every $i$ and $j$. These correspond to assignment to discrete
regions, and the common practice is to assign each individual
to the location that maximises this probability. Each row of the file
contains the data for a single individual. The first two columns give
the individual's id, and the number of loci for which it had genotype
data.  The next two columns give the location numbers of the
true and assigned locations for that individual (the true location
will be -1 if not known).
Subsequent columns give the estimates of the log
probabilities for locations $j=1,2,\dots$.  In version 2.1
these probabilities were in hexadecimal; they are now in decimal.
In general you will want to
use the output from this file {\it only} if you used an
assignment option ({\tt -A} or {\tt -M}).
The file is still produced even if you
don't use an assignment option, but in that case the results are not
based on cross-validation, and assignment results will be
inappropriately overconfident.

\item[{\tt \_params}] Contains sampled values of model parameters, and
log-likelihood of genotype data, for each iteration of the MCMC
(parameters are output during both burnin and main iterations, every
{\tt Nthin} iterations). There are 5 columns, corresponding to sampled
values of $\alpha_0, \alpha_1, \alpha_2$, $\beta$, and the
log-likelihood for the corresponding allele frequency estimates.
This file works well with the Tracer utility (see section \ref{software}) 
which can display plots of the parameter values over time.  This is helpful in diagnosing poor mixing.

\item[{\tt \_accept}] Contains summary of (cumulative) acceptance rates
during MCMC runs. Each line contains data for one iteration of the
MCMC (after thinning). Acceptance rates on each line are, in order,
for $\alpha_1,\alpha_2$ (if these are updated), $X$ and $\mu$.
\item[{\tt \_corr}] Contains estimated and fitted correlations
and covariances between each pair of locations.  The first two columns
give the pair of locations, and the third gives the distance between them
in kilometers inferred from their latitude and longitude.  The following
four columns are empirical and fitted covariance and empirical and
fitted correlation.  ``Empirical" results are estimated from the inferred
allele frequences of the two locations.  ``Fitted" results are estimated
using only the distance between the locations and the inferred model
parameters ($\alpha$) which control the rate at which correlation decays
with distance.  This file may be useful for model checking, specifically
spotting pairs of populations that are much more or less correlated than
expected based on their geographical distance.
\end{enumerate}






%%%%%%%%%%%%%%%%%%%%%%%%%%%%%%%%%%%%%%%%%%%%%%%%%

\section{Sectors and hybrids}

{\bf The documentation, code comments, and code for these options do
not agree and their correctness in the current version is not known.
We do not recommend their use.}

The use of \SCAT for hybrid detection
has been superseded by the {\tt EBhybrids}
program (see software list), and we strongly recommend use of that 
program instead.  It is documented here only for completeness. 
Assignment to sectors can be useful under some circumstances.

Sectors are broad geographic areas containing multiple sampling
locations.  They are most useful when they capture genetic differentiation
not captured by the isolation-by-distance method.

\SCAT can attempt to identify hybrids either between regions or
between sectors.  Unless the regions are genetically
distinct, hybrid identification will be very difficult; hence the
use of sectors.  \SCAT can also attempt to assign locations to
identified hybrids, but this is also problematic.  One may instead
remove identified hybrids from the analysis.

Hybrid testing is turned on with the {\tt -H} option where the H is
followed immediately (no space) by a numeric option.  There are three
meaningful options:

\begin{itemize}
\item -H1 Hybrid analysis with sectors.  The program will estimate,
for each specimen, the probability that it is solely from each of the
sectors, and the probability that it is a hybrid with ancestry from
each pair of sectors.  (Combinations of more than 2 sectors are
not considered.)
\item -H2  Normal analysis with sectors.   This estimates the
sector membership of each elephant, as in the normal region-based
assignment but using sectors instead of regions.  
\item -H11 Hybrid analysis with regions.  The program will estimate 
hybrid probabilities as in H1, but using regions instead of sectors.  
This is unlikely to work unless the regions are genetically distinct.
\end{itemize}

For {\tt -H1} and {\tt -H2} you will need to specify the sector of
each sampling location in the location file, and use the {\tt -Z} option above. 

Sector names must be 0,1,...NSUBREGIONS-1 where NSUBREGIONS is a
constant in the file scat2.hpp, defaulting to 6.  The program {\it may}
work correctly with fewer than NSUBREGIONS sectors, except that 
rows and columns corresponding to those sectors will be meaningless; it
is likely to crash with more than NSUBREGIONS sectors.  To adjust this
number, edit scat2.hpp and recompile.

The results of this analysis are written to a file {\tt Output\_hybrid} in
the output director.  This file contains one row per individual.

For {\tt -H2} there is one column per sector, and the $(i,j)$ entry
is $$\log(\Pr(\text{ind $i$'s genotype data $\mid$ $i$ comes from $j$})).$$
In the case of the ``hybrid'' options ({\tt -H1} and {\tt -H11}) the columns are
$$\log(\Pr(\text{ind $i$'s genotype data $\mid$ $i$ has parents from $(j_1,j_2)$})),$$
for $(j_1,j_2) = (0,0), (0,1),\dots,(0,5),(1,0),(1,1),\dots,(1,5),\dots, (5,5)$.

The maximum entry in a row can be taken as an estimate of the individual's
origin.  For the hybrid cases, the overall probability that an individual is
a hybrid can be taken as the sum of all entries in the row except those indicating
both parent from the same region or sector.

\section{Other software useful in conjunction with \SCAT} \label{software}

These programs are listed as a convenience to the user; we do not
guarantee their availability, compatibility, or performance.

\begin{itemize}
\item {\tt STRUCTURE} ({\tt http://pritch.bsd.uchicago.edu/}) -- estimate
population admixture and assign individuals to populations.
\item {\tt EBhybrids} ({\tt https://github.com/stephenslab/EBhybrids}) -- 
postprocess {\tt STRUCTURE} output to infer hybrid probability for each 
individual.  In our experience, hybrid individuals are difficult for \SCAT 
to assign and it may be best to remove them or at least treat their
inferred locations with skepticism.
\item {\tt TRACER} ({\tt https://beast.community/tracer}) -- display behavior
of an MCMC run over time.  With minimal processing, the {\tt Output\_params}
file can be read into {\tt TRACER} to visualize the progress of the run,
which is useful in diagnosing poor mixing or too-short runs.
\item {\tt VORONOI} ({\tt https://github.com/stephens999/voronoi}) --
postprocess \SCAT estimates to refine continuous assignment on the
assumption that the location-unknown samples may be spatially clustered
relative to the reference samples.
\end{itemize}


\section{Changes to the software} 

I presume that the changes listed in the documentation for version 2.1
(September 8, 2004) were from 2.0 to 2.1; this is not entirely clear in
the documentation.

\subsection{Changes between 2.0 and 2.1}

According to the previous maintainer, ``A number of small changes were made to the MCMC update scheme.
Most of these changes were made to try to
improve mixing for general datasets (rather than the specific one we
analysed)."  

A bug in computation of the likelihood for sectors was fixed.

At some point a change was introduced (probably accidentally) which caused the log probabilities
in file {\tt Output\_probs} to be printed in hexadecimal; it is not known
if this feature was already in 2.0 or was introduced in 2.1.

\subsection{Changes between 2.1 and 2.2}

From this version on, maintenance is being performed by Mary Kuhner
(mkkuhner@uw.edu).

Corrections: The probabilities in file Output\_probs were
changed to decimal; this may disrupt scripts written for the previous
version. 

Improvements:  
Version number was added. 
The Makefile was revised so that "make" makes the \SCAT executable.
Function templates were moved to a file {\tt scat2.hpp}.
The hard-coded limit on number of regions was removed; the program can
now adapt to any reasonable number of regions.
The documentation was extensively rewritten.

Increased error checking of input options and data was added.  Specifically,
each individual is now required to have exactly two haplotypes, and the
number of loci must be the same for all haplotypes of all individuals.

Removals:
Option {\tt -z} which caused certain individuals to be deleted
from the data was removed; it had been added to test a specific hypothesis
on a specific data set and was not suitable for distribution.

%Options to set the maximum number of loci and
%alleles without recompilation, which had been
%present but nonfunctional in the code, were made functional.  

\subsection{Changes between 2.2 and 2.3}

Corrections:  An initialization bug was corrected.  This only disrupted
the first burning step, and likely had little impact on correctness; however,
runs after this fix will not be numerically identical to those before,
as the MCMC search will take a randomly different path.

Internal variable handling was extensively revised to improve maintainability.
This should not impact correctness but may cause the MCMC search to take a
different path due to rounding differences.

\section{How to cite this program}

In publications including results from the use of this program, please
{\it specify the version of the software you used}, and cite:

Wasser SK, Shedlock AM, Comstock K, Ostrander EA, Mutayoba B, Stephens M (2004)
Assigning African elephant DNA to geographic region of origin:  applications
to the ivory trade.  PNAS 101: 14847-14852.


\section{Acknowledgements}

The software makes use of the LAPACK linear algebra routines for
finding the Cholesky decomposition of a matrix, and a version of the
{\tt wn\_PnPoly()} algorithm by Dan Sunday 
$$(\text{\tt http://www.softsurfer.com/Archive/algorithm\_0103/algorithm\_0103.htm})$$.


%\bibliographystyle{chicago}
%\bibliography{/Users/stephens/Dropbox/Documents/mainbib}

\end{document}
